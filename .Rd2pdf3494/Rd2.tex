\documentclass[letterpaper]{book}
\usepackage[times,inconsolata,hyper]{Rd}
\usepackage{makeidx}
\usepackage[latin1]{inputenc} % @SET ENCODING@
% \usepackage{graphicx} % @USE GRAPHICX@
\makeindex{}
\begin{document}
\chapter*{}
\begin{center}
{\textbf{\huge Package matchingMarkets}}
\par\bigskip{\large \today}
\end{center}
\inputencoding{utf8}
\HeaderA{daa}{Deferred Acceptance Algorithm}{daa}
%
\begin{Description}\relax
Finds the student (men) optimal matching in the
\Rhref{http://en.wikipedia.org/wiki/Hospital_resident}{college
admissions}
(\Rhref{http://en.wikipedia.org/wiki/Stable_matching}{stable
marriage}) problem. Uses the Gale-Shapley (1962) Deferred
Acceptance Algorithm with student (male) offer based on
given or randomly generated preferences.
\end{Description}
%
\begin{Usage}
\begin{verbatim}
  daa(nStudents = ncol(s.prefs), nColleges = ncol(c.prefs),
    nSlots = rep(1, nColleges), s.prefs = NULL,
    c.prefs = NULL)
\end{verbatim}
\end{Usage}
%
\begin{Arguments}
\begin{ldescription}
\item[\code{nStudents}] integer indicating the number of
students (in the college admissions problem) or men (in
the stable marriage problem) in the market. Defaults to
\code{ncol(s.prefs)}.

\item[\code{nColleges}] integer indicating the number of
colleges (in the college admissions problem) or women (in
the stable marriage problem) in the market. Defaults to
\code{ncol(c.prefs)}.

\item[\code{nSlots}] vector of length \code{nColleges}
indicating the number of places (i.e. quota) of each
college. Defaults to \code{rep(1,nColleges)} for the
marriage problem.

\item[\code{s.prefs}] matrix of dimension \code{nColleges} x
\code{nStudents} with the \code{i}th column containing
student \code{i}'s ranking over colleges in decreasing
order of preference (i.e. most preferred first).

\item[\code{c.prefs}] matrix of dimension \code{nStudents} x
\code{nColleges} with the \code{j}th column containing
college \code{j}'s ranking over students in decreasing
order of preference (i.e. most preferred first).
\end{ldescription}
\end{Arguments}
%
\begin{Value}
'daa' returns a list with the following items.
\begin{ldescription}
\item[\code{s.prefs}] students' preference matrix.
\item[\code{c.prefs}] colleges' preference matrix.
\item[\code{iterations}] number of interations required to find
the stable matching.\item[\code{matches}] identifier of
students (men) assigned to colleges (women).
\item[\code{match.mat}] matching matrix of dimension
\code{nStudents x nColleges}.\item[\code{singles}] identifier
of single/unmatched students (men).
\end{ldescription}
\end{Value}
%
\begin{Section}{Minimum required arguments}
'daa' requires the following combination of arguments,
subject to the matching problem. \begin{description}

\item[\code{nStudents, nColleges}] Marriage problem with
random preferences.\item[\code{s.prefs,
  c.prefs}] Marriage problem with given preferences.
\item[\code{nStudents, nSlots}] College admissions
problem with random preferences.\item[\code{s.prefs,
  c.prefs, nSlots}] College admissions problem with given
preferences.
\end{description}

\end{Section}
%
\begin{Author}\relax
Thilo Klein \email{thilo@klein.co.uk}
\end{Author}
%
\begin{References}\relax
Gale, D. and Shapley, L.S. (1962). College admissions and
the stability of marriage. The American Mathematical
Monthly, 69(1):9--15.
\end{References}
%
\begin{Examples}
\begin{ExampleCode}
## Marriage problem (3 men, 2 women) with random preferences:
daa(nStudents=3, nColleges=2)

## Marriage problem (3 men, 2 women) with given preferences:
s.prefs <- matrix(c(1,2, 1,2, 1,2), 2,3)
c.prefs <- matrix(c(1,2,3, 1,2,3), 3,2)
daa(s.prefs=s.prefs, c.prefs=c.prefs)

## College admission problem (7 students, 2 colleges
## with 3 slots each) with random preferences:
daa(nStudents=7, nSlots=c(3,3))

## College admission problem (7 students, 2 colleges
## with 3 slots each) with given preferences:
s.prefs <- matrix(c(1,2, 1,2, 1,2, 1,2, 1,2, 1,2, 1,2), 2,7)
c.prefs <- matrix(c(1,2,3,4,5,6,7, 1,2,3,4,5,6,7), 7,2)
daa(s.prefs=s.prefs, c.prefs=c.prefs, nSlots=c(3,3))
\end{ExampleCode}
\end{Examples}
\inputencoding{utf8}
\HeaderA{plp}{Partitioning Linear Programme}{plp}
%
\begin{Description}\relax
Finds the stable matching in the
\Rhref{http://matchingmarkets.org/sites/en.wikipedia.org/wiki/Stable_roommates_problem}{stable
roommates problem} with transferable utility. Uses the
Partitioning Linear Programme formulated in Quint (1991).
\end{Description}
%
\begin{Usage}
\begin{verbatim}
  plp(V = NULL, N = NULL)
\end{verbatim}
\end{Usage}
%
\begin{Arguments}
\begin{ldescription}
\item[\code{N}] number of players in the market

\item[\code{V}] valuation matrix of dimension \code{NxN} that
gives row-players valuation over column players (or vice
versa)
\end{ldescription}
\end{Arguments}
%
\begin{Value}
'plp' returns a list with the following items.
\begin{ldescription}
\item[\code{Assignment.matrix}] upper triangular matrix of
dimension \code{NxN} with entries of 1 for equilibrium
pairs and 0 otherwise.\item[\code{Equilibrium.groups}] matrix
that gives the \code{N/2} equilibrium pairs and
equilibrium partners' mutual valuations.
\end{ldescription}
\end{Value}
%
\begin{Author}\relax
Thilo Klein \email{thilo@klein.co.uk}
\end{Author}
%
\begin{References}\relax
Quint, T. (1991). Necessary and sufficient conditions for
balancedness in partitioning games. Mathematical Social
Sciences, 22(1):87--91.
\end{References}
%
\begin{Examples}
\begin{ExampleCode}
## Roommate problem with 10 players, transferable utility and random preferences:
plp(N=10)

## Roommate problem with 10 players, transferable utility and given preferences:
V <- matrix(rep(1:10, 10), 10, 10)
plp(V=V)
\end{ExampleCode}
\end{Examples}
\inputencoding{utf8}
\HeaderA{ttc}{Top-Trading-Cycles Algorithm}{ttc}
%
\begin{Description}\relax
Finds the stable matching in the
\Rhref{http://en.wikipedia.org/wiki/Herbert_Scarf}{house
allocation problem} with existing tenants. Uses the
Top-Trading-Cycles Algorithm proposed in Abdulkadiroglu
and Sonmez (1999).
\end{Description}
%
\begin{Usage}
\begin{verbatim}
  ttc(P = NULL, X = NULL)
\end{verbatim}
\end{Usage}
%
\begin{Arguments}
\begin{ldescription}
\item[\code{P}] list of individuals' preference rankings over
objects

\item[\code{X}] 2-column-matrix of objects ('obj') and their
owners ('ind')
\end{ldescription}
\end{Arguments}
%
\begin{Value}
'ttc' returns a 2-column matrix of the stable matching
solution for the housing market problem based on the
Top-Trading-Cycles algorithm.
\end{Value}
%
\begin{Author}\relax
Thilo Klein \email{thilo@klein.co.uk}
\end{Author}
%
\begin{References}\relax
Abdulkadiroglu, A. and Sonmez, T. (1999). House
Allocation with Existing Tenants. Journal of Economic
Theory, 88(2):233--260.
\end{References}
%
\begin{Examples}
\begin{ExampleCode}
## generate list of individuals' preference rankings over objects
P <- list()
P[[1]] <- c(2,5,1,4,3)    # individual 1
P[[2]] <- c(1,5,4,3,2)    # individual 2
P[[3]] <- c(2,1,4,3,5)    # individual 3
P[[4]] <- c(2,4,3,1,5)    # individual 4
P[[5]] <- c(4,3,1,2,5); P # individual 5

## generate 2-column-matrix of objects ('obj') and their owners ('ind')
X <- data.frame(ind=1:5, obj=1:5); X

## find assignment based on TTC algorithm
ttc(P=P,X=X)
\end{ExampleCode}
\end{Examples}
